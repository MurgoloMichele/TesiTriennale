\documentclass[11pt,a4paper,twoside,openright]{book}
\usepackage{frontespizio}
\usepackage[english,italian]{babel}
\usepackage{fancyhdr}
\usepackage{listings}
\usepackage[linguistics]{forest}
\usepackage[margin=1in]{geometry}
\usepackage{amsmath}
\usepackage{algorithm}
\usepackage[noend]{algpseudocode}
\usepackage[utf8x]{inputenx}
\usepackage{url}

\makeatletter
\def\BState{\State\hskip-\ALG@thistlm}
\makeatother

\definecolor{folderbg}{RGB}{124,166,198}
\definecolor{folderborder}{RGB}{110,144,169}

\def\Size{4pt}
\tikzset{
  folder/.pic={
    \filldraw[draw=folderborder,top color=folderbg!50,bottom color=folderbg]
      (-1.05*\Size,0.2\Size+5pt) rectangle ++(.75*\Size,-0.2\Size-5pt);  
    \filldraw[draw=folderborder,top color=folderbg!50,bottom color=folderbg]
      (-1.15*\Size,-\Size) rectangle (1.15*\Size,\Size);
  }
}

\newenvironment{abstract}%
	{\cleardoublepage%
		\thispagestyle{empty}%
		\null \vfill\begin{center}%
			\bfseries \abstractname \end{center}}%
	{\vfill\null}

\begin{document}
\begin{frontespizio}
\Universita{Modena}
\Facolta{Scienze Matematiche, Fisiche e Informatiche}
\Corso[Laurea]{Informatica}
\Titoletto{Tesi di laurea}
\Titolo{Titolo tesi}
\Candidato[101851]{Michele Murgolo}
\Relatore{Prof.~Mirco Marchetti}
\Relatore{Prof.~Giovanni Apruzzese}
\Annoaccademico{2017-2018}
\end{frontespizio}

\null\vspace{\stretch{1}}
\begin{flushright}
				\textit{La pagina della dedica}
\end{flushright}
\vspace{\stretch{2}}\null

\cite{Stratosphere}
\begin{abstract}
				Stratosphere Testing Framework (stf) è una framework di ricerca sulla sicurezza della rete per analizzare i modelli comportamentali delle connessioni di rete nel Progetto Stratosphere. Il suo obiettivo è aiutare i ricercatori a trovare nuovi comportamenti malware, etichettare tali comportamenti, creare i loro modelli di traffico e verificare gli algoritmi di rilevamento. Stf funziona utilizzando algoritmi di apprendimento automatico sui modelli comportamentali.
				L'obiettivo di Stratosphere Project è creare un IPS comportamentale (Intrusion Detection System) in grado di rilevare e bloccare i comportamenti dannosi nella rete. Come parte di questo progetto, stf viene utilizzato per generare modelli altamente attendibili di traffico dannoso consentendo una verifica automatica delle prestazioni di rilevamento.
				Il framework genera questi modelli da file in formato binetflow, il DIEF salva il traffico internet in file formato flows.
				Si è scritto un programma in python3 che esegue la conversione batch da flows a binetflow.
				I file che il programma deve convertire sono numerosi e di grandi dimensioni, ogni giorno di traffico ha una dimensione media pari a 150Mb.
				Per effettuare una conversione efficiente si è utilizzato un approccio multicore che ha permesso di ottenere uno speed up lineare della conversione. 
\end{abstract}

\newpage

\chapter{Introduzione}

\section{Preparazione environment}
Per lo sviluppo del programma e lo studio del framework è stata utilizzata una macchina virtuale con Ubuntu 16.04 LTS.

\paragraph{Installazione Stratosphere IPS}
Sulla macchina virtuale è stato installato il framework di Stratospehere IPS. Per l'installazione si sono seguiti i seguenti passaggi:

\begin{itemize}
				\item Installazione del programma git \textit{2.7.4}
\begin{lstlisting}[language=bash]
$ sudo apt install git
\end{lstlisting}

				\item Clonazione repository github del framework
\begin{lstlisting}[language=bash]
$ git clone https://github.com/stratosphereips/StratosphereTestingFramework
\end{lstlisting}

				\item Installazione del programma \textit{python-pip}
\begin{lstlisting}[language=bash]
$ sudo apt install python-pip
\end{lstlisting}
\end{itemize}
				
\begin{itemize}
				\paragraph{Installazione dipendenze per Stratosphere IPS}

				\item prettytable \textit{0.7.2-3}
\begin{lstlisting}[language=bash]
$ sudo apt install python-prettytable
\end{lstlisting}

				\item transaction \textit{1.4.3-3}
\begin{lstlisting}[language=bash]
$ sudo apt install python-transaction
\end{lstlisting}

				\item persistent \textit{4.1.1-1build2}
\begin{lstlisting}[language=bash]
$ sudo apt install python-persistent
\end{lstlisting}

				\item zodb \textit{5.4.0}
\begin{lstlisting}[language=bash]
$ sudo pip install zodb
\end{lstlisting}

				\item sparse \textit{1.1-1.3build1}
\begin{lstlisting}[language=bash]
$ sudo apt install python-sparse
\end{lstlisting}

				\item dateutil \textit{2.4.2-1}
\begin{lstlisting}[language=bash]
$ sudo apt install python-dateutil
\end{lstlisting}

				\item Installazione dell'ultima versione di argus \textit{3.0.8.2} dal sito http://qosient.com/argus/dev/argus-latest.tar.gz

				\item Installazione dell'ultima versione di argus-client \textit{3.0.8.2} dal sito http://qosient.com/argus/dev/argus-clients-latest.tar.gz

\end{itemize}

\paragraph{Per l'installazione di argus di installano le seguenti dipendenze}

\begin{itemize}
				\item libpcap \textit{1.7.4-2}
\begin{lstlisting}[language=bash]
$ sudo apt install libpcab-dev
\end{lstlisting}

\item bison \textit{3.0.4}
\begin{lstlisting}[language=bash]
$ sudo apt install bison
\end{lstlisting}

\item flex \textit{2.6.0-11}
\begin{lstlisting}[language=bash]
$ sudo apt install flex
\end{lstlisting}

\end{itemize}

\paragraph{Utilizzo del programma stf\\}
Per eseguire il programma lo si esegue con
\begin{lstlisting}[language=bash]
	./stf.py
\end{lstlisting}

Per caricare un dataset si utilizza il comando
\begin{lstlisting}[language=bash]
	datasets -c /absolute/path/file.binetflow	
\end{lstlisting}

Per generare la connessione si utilizza il comando
\begin{lstlisting}[language=bash]
	connections -g	
\end{lstlisting}

Infine, per generare i modelli, il comando
\begin{lstlisting}[language=bash]
	models -g	
\end{lstlisting}

Per visualizzare il behavioral model si utilizza il comando
\begin{lstlisting}[language=bash]
	models -L [id]	
\end{lstlisting}

\chapter{Analisi del problema}
I file da convertire hanno una struttura gerarchica\\
\begin{forest}
  for tree={
    font=\ttfamily,
    grow'=0,
    child anchor=west,
    parent anchor=south,
    anchor=west,
    calign=first,
    inner xsep=7pt,
    edge path={
      \noexpand\path [draw, \forestoption{edge}]
      (!u.south west) +(7.5pt,0) |- (.child anchor) pic {folder} \forestoption{edge label};
    },
    before typesetting nodes={
      if n=1
        {insert before={[,phantom]}}
        {}
    },
    fit=band,
    before computing xy={l=15pt},
  }  
[folder structure
  [years
    [months
        [days
            [hours]
        ]
    ]
  ]
]
\end{forest}



//I file all'interno della subdir hours sono sempre 60, uno per ogni minuto e sono compressi con il programma gzip.

I file di tipo flows hanno un header formato da 29 campi

\begin{itemize}

\item{ipv4 source address}
\item{ipv4 destination address}
\item{ipv4 next hop}
\item{input snmp}
\item{output snmp}
\item{input packets}
\item{input bytes}
\item{first switched}
\item{last switched}
\item{source port}
\item{destination port}
\item{tcp flags}
\item{protocol}
\item{source tos}
\item{source as}
\item{destination as}
\item{ipv4 source mask}
\item{ipv4 destination mask}
\item{l7 protocol}
\item{biflow direction}
\item{flow start seconds}
\item{flow end seconds}
\item{output packets}
\item{output bytes}
\item{flow id}
\item{flow active timeout}
\item{flow inactive timeout}
\item{input source mac}
\item{output destination mac}

\end{itemize}

Mentre nei file binetflow l'header è formato dai seguenti campi

\begin{itemize}

\item{start time}
\item{duration}
\item{protocol}
\item{source address}
\item{source port}
\item{direction}
\item{destination address}
\item{destination port}
\item{state}
\item{source tos}
\item{destination tos}
\item{tot packets}
\item{tot bytes}
\item{source bytes}
\item{source data}
\item{destination data}
\item{label}

\end{itemize}

I campi del file flows che non compaiono in quelli del file binetflow sono stati eliminati. 

\begin{table}[]
				\centering
				\caption{Tabella di conversione}
				\label{Tabella di conversione}
				\begin{tabular}{|l|l|}
								\hline
								binetflow												 & flow																								 \\ \hline
								start time											 & first switched																			 \\ \hline
								duration												 & last switched - first switched											 \\ \hline
								protocol												 & protocol																						 \\ \hline
								source address                   & ipv4 source address                                 \\ \hline
								source port                      & source port                                         \\ \hline
								direction                        & biflow direction                                    \\ \hline
								destination address              & ipv4 destination address                            \\ \hline
								destination port                 & destination port                                    \\ \hline
								state                            & -                                                   \\ \hline
								source tos                       & source tos                                          \\ \hline
								destination tos                  & -                                                   \\ \hline
								tot packets                      & input packets + output packets                      \\ \hline
								tot bytes                        & input bytes + output bytes                          \\ \hline
								source bytes                     & input bytes                                         \\ \hline
								source data											 & -																									 \\ \hline
								destination data								 & -																									 \\ \hline
								label														 & -																									 \\ \hline

				\end{tabular}
\end{table}

\chapter{Soluzione proposta}
La scrittura del programma è stata effettuata in due fasi, nella prima è stato realizzato un programma che effettua la conversione in single core. Nella seconda fase si è passati alla parallelizzazione del programma.


\section{Versione single core}
In questa versione il programma effettua le letture, le conversioni e le scritture su file con un unico processore.

Pseudocodice del programma single core
\begin{algorithm}
\caption{Single core version}\label{single core version}
\begin{algorithmic}[1]
\Procedure{Hydra}{}
\ForAll{\textit{file} in path}
\State read data from \textit{file}
\State convert data into new format
\State append data into new file
\EndFor

\EndProcedure
\end{algorithmic}
\end{algorithm}
\section{Versione multi core}

Pseudocodice del programma multi core
\begin{algorithm}
\caption{Multi core version}\label{multi core version}
\begin{algorithmic}[1]
\Procedure{Hydra}{}
\ForAll{\textit{file} in path}
\State $\textit{Queue[]} \gets \textit{file}$
\State read data from $\textit{file}$
\State spawn 4 process
\While{\textit{Queue[]} $\textbf{not empty}$}
\State $\textit{filename} \gets $\textit{Queue.get()}
\State convert data into new format
\State append data into new file
\EndWhile
\EndFor
\EndProcedure
\end{algorithmic}
\end{algorithm}


\chapter{Esperimenti e risultati}

\section{benchmark single core}

I benchmark sono stati effettuati in due modalità:
\begin{enumerate}
				\item cold cache
				\item warm cache
\end{enumerate}

Sono stati effettuati 10 test in entrambe le modalità. I risultati vengono riportati a seguire.//
tutti i benchmark sono stati effettuati con \textit{/usr/bin/time}

\paragraph{warm cache benchmarks}
\begin{enumerate}
				\item 2:28:03 98\%CPU
				\item 2:27:98 98\%CPU
				\item 2:28:86 98\%CPU
				\item 2:28:97 95\%CPU
				\item 2:24:52 99\%CPU
				\item 2:34:21 94\%CPU
				\item 2:28:36 98\%CPU
				\item 2:35:02 93\%CPU
				\item 2:25:59 99\%CPU
				\item 2:28:75 96\%CPU
\end{enumerate}

La media calcolata è quindi di \textbf{2:28:38 97\%CPU}

\paragraph{cold cache benchmarks}
\begin{enumerate}
				\item 2:30:92 95\%CPU
				\item 2:32:67 94\%CPU
				\item 2:25:34 99\%CPU
				\item 2:28:13 97\%CPU
				\item 2:30:27 95\%CPU
				\item 2:26:01 98\%CPU
				\item 2:28:29 96\%CPU
				\item 2:26:79 98\%CPU
				\item 2:27:97 97\%CPU
				\item 2:31:14 97\%CPU
\end{enumerate}

La media calcolata è quindi di \textbf{2:28:75 97\%CPU}

Come è possibile notare i risultati non variano in modo apprezzabile.


\section{Benchmark multi core}
I benchmark sono stati effettuati in due modalità:
\begin{enumerate}
				\item cold cache
				\item warm cache
\end{enumerate}
Sono stati effettuati 10 test in entrambe le modalità. I risultati vengono riportati a seguire.//
Tutti i benchmark sono stati effettuati con \textit{/usr/bin/time}

\paragraph{warm cache benchmarks}
\begin{enumerate}
				\item 0:54:82 265\%CPU
				\item 0:36:13 375\%CPU
				\item 0:38:24 367\%CPU
				\item 0:40:25 377\%CPU
				\item 0:40:14 368\%CPU
				\item 0:44:28 351\%CPU
				\item 0:41:67 368\%CPU
				\item 0:40:72 383\%CPU
				\item 0:40:83 381\%CPU
				\item 0:41:19 382\%CPU
\end{enumerate}
La media calcolata è quindi di \textbf{0:41:83 362\%CPU}

\paragraph{cold cache benchmarks}
\begin{enumerate}
				\item 0:35:12 383\%CPU
				\item 0:36:61 370\%CPU
				\item 0:38:21 379\%CPU
				\item 0:34:76 388\%CPU
				\item 0:38:58 375\%CPU
				\item 0:36:28 377\%CPU
				\item 0:41:48 358\%CPU
				\item 0:43:31 340\%CPU
				\item 0:41:01 376\%CPU
				\item 0:36:50 377\%CPU
\end{enumerate}
La media calcolata è quindi di \textbf{0:38:19 372\%CPU}

\chapter{Conclusioni}
\section{Prestazioni}

Lo \textit{speedup relativo} su \textit{p} processori si calcola come

\begin{center}
\begin{math}
S(p) = \frac{T(1)}{T(p)}
\end{math}
\end{center}

Con i risultati ottenuti si ha uno \textit{speedup relativo} di

\begin{center}
\begin{math}
S(p) = \frac{T(148)}{T(40)} = \textbf{3,7}
\end{math}
\end{center}

In un sistema ideale, in cui il carico di lavoro potrebbe essere perfettamente partizionato su \textit{p} processori, lo speedup relativo dovrebbe essere uguale a p. In questo caso si parla di \textbf{speedup lineare}.

Si definisce \textit{efficienza} il rapporto
\begin{center}
\begin{math}
E(p) = \frac{S(p)}{p}
\end{math}
\end{center}

Idealmente, se l'algoritmo avesse uno speedup lineare, si avrebbe 
\begin{math}
E(p) = 1
\end{math}

Più l'efficienza si allontana da 1, peggio stiamo sfruttando le risorse di calcolo disponibili nel sistema parallelo.
\begin{center}
\begin{math}
				E(p) = \frac{S(3,7)}{4} = \textbf{0,925}
\end{math}
\end{center}

\end{document}
