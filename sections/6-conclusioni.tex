\documentclass[../main.tex]{subfiles}
\begin{document}
\chapter{Conclusioni}
\section{Prestazioni}

Sia \textit{T(p)} il tempo di esecuzione in secondi di un certo algoritmo su \textit{p processori}. Di conseguenza sia \textit{T(1)} il tempo di esecuzione del codice parallelo su 1 processore.
La \textit{misura di scalabilità} o \textit{speedup} relativo di un algoritmo parallelo eseguito su \textit{p} processori si calcola come:

\begin{center}
\begin{math}
S(p) = \frac{T(1)}{T(p)}
\end{math}
\end{center}

Con i risultati ottenuti si ha uno \textit{speedup relativo} di

\begin{center}
\begin{math}
S(p) = \frac{T(148)}{T(40)} = \textbf{3,7}
\end{math}
\end{center}

In un sistema ideale, in cui il carico di lavoro potrebbe essere perfettamente partizionato su \textit{p} processori, lo speedup relativo dovrebbe essere uguale a p. In questo caso si parla di \textbf{speedup lineare}.

Si definisce \textit{efficienza} il rapporto
\begin{center}
\begin{math}
E(p) = \frac{S(p)}{p}
\end{math}
\end{center}

Idealmente, se l'algoritmo avesse uno speedup lineare, si avrebbe 
\begin{math}
E(p) = 1
\end{math}

Più l'efficienza si allontana da 1, peggio stiamo sfruttando le risorse di calcolo disponibili nel sistema parallelo.
\begin{center}
\begin{math}
				E(p) = \frac{S(3,7)}{4} = \textbf{0,925}
\end{math}
\end{center}
\end{document}
