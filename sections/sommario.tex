\documentclass[../main.tex]{subfiles}

\begin{document}
\cite{Stratosphere}
\begin{abstract}
				Stratosphere Testing Framework (stf) è una framework di ricerca sulla sicurezza della rete per analizzare i modelli comportamentali delle connessioni di rete nel Progetto Stratosphere. Il suo obiettivo è aiutare i ricercatori a trovare nuovi comportamenti malware, etichettare tali comportamenti, creare i loro modelli di traffico e verificare gli algoritmi di rilevamento. Stf funziona utilizzando algoritmi di apprendimento automatico sui modelli comportamentali.
				L'obiettivo di Stratosphere Project è creare un IPS comportamentale (Intrusion Detection System) in grado di rilevare e bloccare i comportamenti dannosi nella rete. Come parte di questo progetto, stf viene utilizzato per generare modelli altamente attendibili di traffico dannoso consentendo una verifica automatica delle prestazioni di rilevamento.
				Il framework genera questi modelli da file in formato binetflow, il DIEF salva il traffico internet in file formato flows.
				Si è scritto un programma in python3 che esegue la conversione batch da flows a binetflow.
				I file che il programma deve convertire sono numerosi e di grandi dimensioni, ogni giorno di traffico ha una dimensione media pari a 150Mb.
				Per effettuare una conversione efficiente si è utilizzato un approccio multicore che ha permesso di ottenere uno speed up lineare della conversione. 
\end{abstract}
\end{document}
