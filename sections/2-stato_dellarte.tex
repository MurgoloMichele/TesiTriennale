\documentclass[../main.tex]{subfiles}
\begin{document}
\chapter{Stato dell'arte}

\textit{Network Intrusion Detection System (IDS)} è un sistema di rilevamento delle intrusioni: una tecnologia di sicurezza efficace, che può rilevare, prevenire e possibilmente reagire agli attacchi informatici, uno dei componenti standard nelle infrastrutture di sicurezza. Monitora le fonti di attività del traffico della rete e distribuisce varie tecniche per fornire servizi di sicurezza. L'obiettivo principale degli \textit{IDS} è quello di rilevare tutte le intrusioni in modo efficiente, l'implementazione consente agli amministratori di rete di rilevare violazioni degli obiettivi di sicurezza.

Esistono diversi tipi di tecniche per rilevare le intrusioni. In questa tesi si è fatto uso di un \textit{IDS} che utilizza algoritmi di \textit{machine learning}.

\textit{Machine learning} può essere definito come la capacità di un programma per computer di apprendere e migliorare le prestazioni su una serie di attività nel tempo. Le tecniche di \textit{machine learning} si concentrano sulla costruzione di un modello, \textit{behavioral model}, di sistema che migliora le sue prestazioni in base ai risultati precedenti. \newline

Una \textit{botnet} è una rete di computer compromessi sotto il controllo di un attore malintenzionato. Un bot si forma quando un computer viene infettato da un malware che ne consente il controllo da terze parti. I computer infetti sono noti anche come \textit{zombie} per la loro capacità di operare in direzione remota senza la conoscenza dei loro proprietari. Negli ultimi anni il \textit{botnet detection} è stato un tema caldo a causa dell'aumento dell'attività malevola.

L'utilizzo di \textit{IDS} è un approccio utile per fare \textit{botnet detection} sul traffico di rete, si osserva il traffico di dati nella rete e si cercano comunicazioni sospette che possono essere fornite da \textit{bot}. \newline

\end{document}
