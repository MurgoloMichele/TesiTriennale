\documentclass[../main.tex]{subfiles}
\begin{document}
\chapter{Analisi del problema}

\section{Network flows}
Le misurazioni del traffico sono necessarie per gestire tutti i tipi di reti IP. L'amministratore delle reti ha bisogno di una vista dettagliata del traffico di rete per ragioni di sicurezza, contabilità e gestione. Le composizioni del traffico devono essere analizzate accuratamente quando si valutano le metriche del traffico o quando si riscontrano problemi di rete.

Tutte queste misurazioni devono essere fatte analizzando tutti i pacchetti che scorrono verso i punti centrali della rete come router o switch. L'analisi di questi pacchetti può essere eseguita all'istante o registrando tutti i pacchetti e analizzarli in un secondo momento. Con l'aumento delle capacità di rete e dei volumi di traffico questo tipo di approccio non è molto efficiente. 

Invece i pacchetti simili (pacchetti con un insieme di proprietà comuni) possono essere raggruppati insieme per la composizioni di \textit{flows}. Un \textit{flow} può essere composto da tutti i pacchetti che condividono la stessa origine e l'indirizzo di destinazione, in questo modo tipi simili di traffico possono essere memorizzati in un formato più compatto senza perdere le informazioni a cui siamo interessati.

Una volta raccolte queste informazioni si ha una vista dettagliata del traffico di rete. 

\section{Argus}

\textit{Audit Record Generation and Utilization System} è la prima implementazione del monitoraggio dei flussi di rete ed è un progetto di monitoraggio continuo del flusso di rete open source. \newline

Questo programma è utilizzato da \textit{Stratosphere IPS} e di conseguenza accetta in input file che rispettano il formato di \textit{Argus}. Questi file hanno estensione \textit{*.binetflow} e hanno un header formato da 17 campi, separati dal carattere ","
\begin{table}[H]
\begin{tabular}{|l|l|}
\hline
\textbf{campo} & \textbf{descrizione}                    \\ \hline
StartTime      & record start time                       \\ \hline
Dur            & record total duration                   \\ \hline
Proto          & transaction protocol                    \\ \hline
SrcAddr        & source IP address                       \\ \hline
Sport          & source port number                      \\ \hline
Dir            & direction of transaction                \\ \hline
DstAddr        & destination IP address                  \\ \hline
Dport          & destination port number                 \\ \hline
State          & transaction state                       \\ \hline
sTos           & source TOS byte value                   \\ \hline
dTos           & destination TOS byte value              \\ \hline
TotPkts        & total transaction packet count          \\ \hline
TotBytes       & total transaction bytes                 \\ \hline
SrcBytes       & src -\textgreater dst transaction bytes \\ \hline
srcUdata       & source user data buffer                 \\ \hline
dstUdata       & destination user data buffer            \\ \hline
Label          & metadata label                          \\ \hline
\end{tabular}
\end{table}

\section{nProbe}

Negli ambienti commerciali, \textit{NetFlow} è probabilmente lo standard \textit{de facto} per la contabilità e la fatturazione del traffico di rete. \textit{NetFlow} è una tecnologia creata originariamente da Cisco e ora è standardizzata come \textit{Internet Protocol Flow Information eXport} \newline
Il \textit{DIEF} utilizza \textit{nProbe} per la cattura e l'analisi dei pacchetti di reti. La cattura dei pacchetti viene salvata su file che hanno formato proprietario, con estensione \textit{*.flows}.
Questo formato ha un header formato da 29 campi, ciascun campo è separato da un carattere speciale.

\begin{table}[H]
\begin{tabular}{|l|l|}
\hline
\multicolumn{1}{|c|}{\textbf{campo}} & \multicolumn{1}{c|}{\textbf{descrizione}}     \\ \hline
IPV4\_SRC\_ADDR                      & IPv4 source address                           \\ \hline
IPV4\_DST\_ADDR                      & IPv4 destination address                      \\ \hline
IPV4\_NEXT\_HOP                      & IPv4 next hop address                         \\ \hline
INPUT\_SNMP                          & input interface SNMP idx                      \\ \hline
OUTPUT\_SNMP                         & output interface SNMP idx                     \\ \hline
IN\_PKTS                             & incoming flow packets (src -\textgreater dst) \\ \hline
IN\_BYTES                            & incoming flow bytes (src -\textgreater dst)   \\ \hline
FIRST\_SWITCHED                      & SysUptime (msec) of the first flow pkt        \\ \hline
LAST\_SWITCHED                       & SysUptime (msec) of the last flow pkt         \\ \hline
L4\_SRC\_PORT                        & IPv4 source port                              \\ \hline
L4\_DST\_PORT                        & IPv4 destination port                         \\ \hline
TCP\_FLAGS                           & cumulative of all flow TCP flags              \\ \hline
PROTOCOL                             & IP protocol byte                              \\ \hline
SRC\_TOS                             & Type of service byte                          \\ \hline
SRC\_AS                              & source BGP AS                                 \\ \hline
DST\_AS                              & destination BGP AS                            \\ \hline
IPV4\_SRC\_MASK                      & IPv4 source subnet mask                       \\ \hline
IPV4\_DST\_MASK                      & IPv4 dest subnet mask                         \\ \hline
L7\_PROTO                            & layer 7 protocol (numeric)                    \\ \hline
BIFLOW\_DIRECTION                    & 1=initiator, 2=reverseInitiator               \\ \hline
FLOW\_START\_SEC                     & seconds (epoch) of the first flow packet      \\ \hline
FLOW\_END\_SEC                       & seconds (epoch) of the last flow packet       \\ \hline
OUT\_PKTS                            & outgoing flow packets (dst -\textgreater src) \\ \hline
OUT\_BYTES                           & outgoing flow bytes (dst -\textgreater src)   \\ \hline
FLOW\_ID                             & serial flow identifier                        \\ \hline
FLOW\_ACTIVE\_TIMEOUT                & activity timeout of flow cache entries        \\ \hline
FLOW\_INACTIVE\_TIMEOUT              & inactivity timeout of flow cache entries      \\ \hline
IN\_SRC\_MAC                         & source MAC address                            \\ \hline
OUT\_DST\_MAC                        & destination MAC address                       \\ \hline
\end{tabular}
\end{table}

\section{Stratosphere IPS}
In questa tesi si è utilizzato Stratosphere Testing Framework, un \textit{Network Intrusion Detection System} che genera modelli comportamentali delle connessioni di reti. Il suo obiettivo è aiutare i ricercatori a trovare nuovi comportamenti malware, etichettare tali comportamenti, creare i loro modelli di traffico e verificare gli algoritmi di rilevamento. Stf funziona utilizzando algoritmi di apprendimento automatico sui modelli comportamentali.

L'obiettivo di Stratosphere Project è quello di creare un \textit{IDS} comportamentale in grado di rilevare e bloccare i comportamenti dannosi nella rete.

Come parte di questo progetto, stf viene utilizzato per generare modelli altamente attendibili di traffico dannoso consentendo una verifica automatica delle prestazioni di rilevamento. \newline

\textit{Stratosphere IPS} non è strettamente un \textit{IPS} nel senso che può impedire l'intrusione. Usa l'acronimo \textit{IPS} perchè l'\textit{IPS} di Stratosphere può bloccare connessioni malevoli usando il firewall del computer. Tuttavia, a causa della natura delle connessioni di traffico, l'\textit{IPS} di Stratosphere necessita di un po' di tempo per rilevare il comportamento dannoso e quindi non può bloccare i primi pacchetti nella connessione.

L'\textit{IPS} di Stratosphere è in grado di rilevare e bloccare connessioni di rete molto fini e pericolose, e quindi dovrebbe essere visto come un complemento delle attuali misure di sicurezza della rete.

\subsection{Il significato dei modelli comportamentali} 
Il nucleo di \textit{Stratosphere IPS} è composto dai modelli comportamentali di reti e algoritmi di rilevamento. I modelli comportamentali rappresentano ciò che una connessione specifica fa nella rete durante la sua vita. Il comportamento è costuito analizzando la sua periodicità, le dimensioni e la durata di ciascun flusso. Sulla base di queste caratteristiche a ciascun flusso viene assegnata una lettera e il gruppo di lettere caratterizza il comportamento della connessione.

Prendiamo come esempio una connessione generata da una botnet che ha il seguente modello comportamentale
\begin{lstlisting}[language=bash]
88*y*y*i*H*H*H*y*0yy*H*H*H*y*y*y*y*H*h*y*h*h*H*H*h*H*y*y*y*H*
\end{lstlisting}

Questa catena di stati che chiamiamo modello comportamentale evidenzia alcune delle caratteristiche del canale C\&C. In questo caso ci dice che i flussi sono altamente periodici (lettere \textit{h}, \textit{i}), con qualche periodicità persa vicino all'inizio (lettere \textit{y}). I flussi hanno anche una grande dimensione con una durata media. I simboli tra le lettere sono correlati al tempo trascorso tra i flussi. In questo caso il simbolo '\textit{*}' significa che il flusso è separato da meno di un'ora. Guardando le lettere si può vedere che questa è una connessione piuttosto periodica, e controllando efficacemente i suoi flussi confermiamo tale ipotesi.
Con l'utilizzo di questo tipo di modelli siamo in grado di generare le caratteristiche comportamentali di un gran numero di azioni dannose. L'immagine seguente mostra i criteri di assegnazione delle lettere per i modelli comportamentali \newline
\includegraphics[scale=0.5]{modello-comportamentale.png}


Gli algoritmi di rilevamento utilizzano modelli comportamentali proprietari dannosi per rilevare nuove connessioni sospette nella rete. Il rilevamento viene attualmente eseguito utilizzando algoritmi basati su catene di \textit{Markov}.
La prima parte dell'algoritmo consiste nell'apprendimento e nell'etichettatura del traffico di verità di base. Questo traffico viene utilizzato per creare modelli verificati di comportamenti di rete noti e stabili. \newline

La seconda parte dell'algoritmo consiste nell'utilizzare questi modelli di verità di base noti e verificati per rilevare comportamenti simili in reti sconosciute. \textit{Stratosphere IPS} catturerà il traffico in un computer client e confronterà ogni connessione sconosciuta con i modelli conosciuti di comportamento del traffico. Poichè il modo in cui viene effettuato il rilevamento e come vengono creati i modelli, ciascun modello comportamentale può corrispondere a un'ampia gamma di comportamenti simili senza essere troppo generico. I modelli sono quindi utili per trovare comportamenti simili senza il rischio di generare troppi falsi positivi.


\section{Problematiche dovute all'utilizzo di due diversi formati}
Come si è potuto notare nelle sezioni \textit{2.2} e \textit{2.3}, gli header di \textit{nProbe} ed \textit{Argus} presentano delle differenze che non permettono di essere utilizzati in modo intercambiale. I file che produce in output \textit{Argus} presentano 17 cambi, ben 12 in meno rispetto ai file di output prodotti da \textit{nProbe}.
L'utilizzazione di due diversi formati presenta errori quando si cerca di utilizzare file prodotti da \textit{nProbe} in Stratosphere IPS. \newline

Questa incompatibilità ha portato alla necessità di una conversione: i file prodotti dal \textit{DIEF} devono essere convertiti in file di formato usato da \textit{Argus}. Questa conversione deve essere efficiente e precisa.

\section{Presentazione del problema}
Il \textit{DIEF} salva i file in una struttura gerarchica fissa ben definita: ci sono 4 livelli di subdir, in cui il primo livello indica l'anno, il secondo il mese, il terzo il giorno e il quarto l'ora. All'interno dell'ultima subdir, quella delle ore, ci sono 60 file uno per ogni minuto della giornata.

\begin{forest}
  for tree={
    font=\ttfamily,
    grow'=0,
    child anchor=west,
    parent anchor=south,
    anchor=west,
    calign=first,
    inner xsep=7pt,
    edge path={
      \noexpand\path [draw, \forestoption{edge}]
      (!u.south west) +(7.5pt,0) |- (.child anchor) pic {folder} \forestoption{edge label};
    },
    before typesetting nodes={
      if n=1
        {insert before={[,phantom]}}
        {}
    },
    fit=band,
    before computing xy={l=15pt},
  }  
[folder structure
  [years
    [months
        [days
            [hours]
        ]
    ]
  ]
]
\end{forest} \newline \newline

I file dei minuti sono compressi usando il programma \textit{gzip}, pertanto c'è da tenerne conto nella soluzione per la conversione.
\end{document}
